\documentclass{article}
\usepackage[utf8]{inputenc}
\usepackage[english]{babel}

\usepackage{amsfonts}
\usepackage{amsmath}
\usepackage{amsthm}
\usepackage{xcolor}

\DeclareMathOperator\erf{erf}


\usepackage[margin=0.7in]{geometry}



\title{Quick Notes on Probabilisitic Peridynamics}
\author{Tim}
\date{\today}

\begin{document}

\maketitle


\begin{itemize}

\section{Introduction}

\item Peridynamics is a reformulation of classical mechanics, which gives rise to a nonlocal continuum model.

\item The key feature is formulation is written as integral representation of the force density around any material point. Consequently there are no assumptions of the differentiability of the displacement field. Hence, it's significant advantage over classical continuum models built around weak-derivatives of the stress tensor. Peridynamic naturally handles the discontinuities which arise when modelling fracture mechanics.

\item Central idea of this work, is to reformulate the discretised peridyanmics equations by adding Brownian motion to each particle. Thus allowing us to capture our uncertainty in capture the material failure at the macroscale, from missing information at the microscale. This additive Brownian forcing, can be seen as a bias corrector of the model, for which if we had data we could infer. The output would be a data-adjusted peridynamic formulation.

\item There is a connection to O'Hagan work on model mis-specification, yet the key difference is that the bias correction is not in the data/observation space, but applied to the process itself. We believe this is potentially a much richer formulation.

\item For engineers we see two key outputs
\begin{itemize}
\item Whilst deterministic peridynamics has shown great potential in model the evolution of cracks the biggest complaints in the literature are that it is difficult / not clear how to chose the additional parameters (such as the characteristic length scale or material horizon $\delta$). This is a common complaint in higher-order continuum models, Cosserat models are a good example. This motivates a Bayesian parameter estimation approach.

\item Current deterministic models, give a single crack evolution. By sampling the model for different realisation of Brownian noise, we can obtain a distribution of crack paths. This gives a much richer engineering output, since it safe guards again near by low energy states, in which the cracks may evolve at lower design loads.

\end{itemize}


\end{itemize}



\section{Probabilistic Peridynamics}

In this section, we start by recapping the key concepts of the deterministic peridynamic formulation, and gradually introduce the reformulation into a probabilistic setting. Here the formulation presented is immediately the discretised version of the peridynamic model. The formulation as a integral partial differential equation, prior to adopting a solution strategy is well documented, see for example [{\color{blue}Host of references - populating Git 'paper' folder}].

\smallskip
Let a body in it's initial configuration occupy a domain $\mathcal B \subset \mathbb R^d$. In peridyanmics, computaionally the material can be represented by $N$ particles, each with the initial coordinates ${\bf x}_i \in \mathcal B$, displacement vector ${\bf u}_i(t) \in [0,T] \times \mathbb R^d$ and associated lumped volumes $dV_i$. Furthermore a given particle is subject to an externally applied force ${\bf p}_i \in \mathbb R^d$. At any point in time $t$ the position of a particle is
$$
{\bf y}_i(t) = {\bf x}_i + {\bf u}_i(t) \quad \mbox{for} \quad t > 0.
$$
The peridynamic formulation is a {\em non-local} model, since it accounts for interactions between material points at finite distance. Therefore, the $i^{th}$ particle interacts with the all particles within the index set
$$
\mathcal H_i := \{ j \; | \; \| \textbf{x}_{(i)} - \textbf{x}_{(j)} \|< \delta \}.
$$
The user defined parameter $\delta \geq 0$ is the {\em horizon}, which describes a characteristic length scale specific to the mechanics of the material.

\smallskip
In this contribution a bond-based peridynamic formulation is considered [Silling2008]. The approach outlined is not limited to this constraint, and the re-formulation of state-based peridynamics approach into a probabilistic framework would follow the steps outline below. Here, we enjoy the relative simplicity of bond-based models, and focus our attention of exploring the novel probabilistic elements of the new formulation.  Therefore, between a pair of particles $i$ and $j$, which reside in each others horizons, a non-linear connection define by the strain energy density $w_{ij}({\bf u}, \theta) > 0$ is made, where $\theta$ are model parameters. Each of which a described in the paragraphs which follow. The derivative $\nabla_{{\bf u}_i} w_{ij} = {\bf f}_{ij}$, gives the force vector ${\bf f}_{ij}$ the particle $i$ receives due to its interaction with particle $j$.

\noindent
The connection between two particles has an original length of
\begin{equation}
\ell_{ij} = \ell_{ji} = \| \textbf{x}^{(i)} - \textbf{x}^{(j)} \|_2
\end{equation}
at which the connection stores no strain energy. Under a general deformation of each particle the strain of the connection (stretch over original length) is given by
$$
\varepsilon_{ij} = \frac{\| \textbf{y}_j - \textbf{y}_i\|_2 - \ell_{ij}}{\ell_{ij}}.
$$
We define the strain /  density curve project in the axial direction of the bond $\boldsymbol{\xi} = \textbf{y}_j - \textbf{y}_i$, by
\begin{equation}\label{eqn:simpleBond}
f = \boldsymbol{\xi} \cdot \nabla_{{\bf u}_i}w_{ij}({\bf u},\theta) :=
\begin{cases}
 c \varepsilon_{ij} \quad \mbox{for} \quad \varepsilon_{ij} < s_{00} \; \mbox{and} \; \max_{t' \in [0,t]} \varepsilon_{ij}(t') < s_{00}  \\
 0 \quad \mbox{otherwise}
\end{cases}
\end{equation}
Here the constant $c$ defines the peridynamic spring stiffness. For simple bond based peridynamic models [Silling2008], this is defined as
$$
c = \frac{18 K}{\pi \delta^4}
$$
for some material bulk modulus $K$ and horizon $\delta$. The other new parameter introduced in \eqref{eqn:simpleBond} is $s_{00}$. This defines a {\em critical strain} of the bond, once exceed the bond reversibily breaks.

{\em Remark:} A simple smooth approximation to this behaviour can be constructed to ensure that $f \in C^1$ (or $w_{ij} \in C^2$) if required by theory to define an invariant measure of the results SDE. Such a smooth approximation has no practical difference to the physical results, from a modelling perspective.

\smallskip
For a general configuration ${\bf u}$, the strain energy stored within the body $\mathcal B$, is given by
\begin{equation}\label{eqn:strainEnergy}
V({\bf u},\theta) = \sum_{i=1}^N \left[\left(\frac{1}{2}\sum_{j \in \mathcal H_i} w_{ij}({\bf u},\theta)dV_j \right) - {\bf p}_i^T{\bf u}_i\right] \left( + \sum_{i=1}^N \frac{1}{2}\epsilon\; {\bf u}_i^T{\bf u}_{i} \right)
\end{equation}
The first term is the sum of (half the) strain energy in all connecting bonds with a given particle. The half, simply shares the energy equally between the two particles which make up the bond, preventing any double accounting. The second term is the work done by an external load $\bf{p}_i$ acting on a particle. The final term, is an additional term with confines the problem, energetically penalising extremely large displacements in the physical case where a particle is completely de-bonded from all others. The parameter controlling the influence of this final term $\epsilon > 0$, can be chosen to be arbitrarily small. This additional term, over standard peridynamics will be required to ensure that the formulation of the probabilistic peridynamics SDE has an invariant measure.

\smallskip
A model for the evolution of the elastic body, is to assume that the system evolves to a configuration which minimises the total potential energy $V({\bf u})$. It is therefore natural to define the deterministic gradient flow
$$
\eta \;\partial_t \textbf{u} = -\nabla V ({\bf u}) \quad \mbox{for} \quad t > 0 \quad \mbox{and} \quad \textbf{U}(0) \quad \mbox{given}.
$$
Where $\textbf{U} = [\textbf{u}_1,\ldots,\textbf{u}_n]^T$, is the complete solution vector, and $\eta$ is a parameter which has the physical interpretation as the damping/viscosity of the material. The value of $\eta$ is a measure of the responsiveness of the material to non-equilibrium in forces acting on it. In this contribution the effects of inertia are neglected, but could in general be included by defining the new variable.

\smallskip\noindent
The approach set out in this paper is to reformulate the deterministic equation as a Stochastic Differential Equation (SDE), we derived the equation for overdamped Langevin dynamics, given by

\smallskip
\begin{equation}\label{eqn:sde}
d\textbf{u} = -\eta^{-1}\mathcal{C}\nabla V (\textbf{u})dt  + \sqrt{2\mathcal C}\;d\textbf{B} \quad \mbox{for} \quad t > 0 \quad \mbox{and} \quad \textbf{u}(0) \quad \mbox{given}.
\end{equation}
where $\textbf{B}$ denotes $N$-dimensional Brownian motion in $\mathbb R^d$ and $\mathcal C$ defines a covariance function

\smallskip
{\color{blue}
\noindent{\bf The current state  of play:}

\begin{itemize}
\item We have nice simple example, where we can show the stochastic evolution of a crack over time, and resulting distribution of a family of crack paths.

\item We notice that the formulation is not a classical overdampled Langevin, once the material states to break. The $V({\bf u})$ becomes dependent on it's history. Therefore the analysis becomes much more complex.

\item Up to the point of failure with a smooth approximation of \eqref{eqn:simpleBond} and the additional constraining term in the potential energy \eqref{eqn:strainEnergy}, we believe it is straight forward to show that the SDE has an invariant measure.

\item Clearly the bit that is of interest is when the material starts to fail.
\end{itemize}

}


\newpage 

\subsection{Potential Reformulation of Peridynamics to allow memory gradient effect}

For each bond, which connects particles $i$ and $j$ we introduce the new parameter $\alpha_{ij}$. This measures the damage ($1 - \alpha_{ij}$)  in the bond, such that $\alpha_{ij} = 1$ means there is no damage, and if $\alpha_{ij} = 0$ the bond is fully damaged. We assume the evolution of the damage obeys the differential equation
$$
\frac{d}{dt}\left(\alpha_{ij}\right) = -\exp\left(k\varepsilon_{ij}\right)\alpha_{ij}^{n}(1-\alpha_{ij})^{m} \quad \mbox{where} \quad t \geq 0 \quad \mbox{and} \quad 0 \leq \alpha_{ij}(0) < 1 .
$$
Material parameters $k$, $\lambda$, $n$ and $m$ all positive constants and $\varepsilon_{ij}$ is axial strain in the bond.
%$$
%\tilde{k} = \frac{1}{2} \sqrt{\pi\lambda}\left( \erf\left(\frac{\epsilon^\star}{\sqrt{\lambda}}\right) + 1\right)
%$$ 
\begin{itemize}
\item The damage $1 - \alpha_{ij}$ is strictly increasing, and bounded above by $1$.

\item $\alpha_{ij} = \alpha_{ji}$

\item Note $\alpha_{ij}(0) < 1$, otherwise you get no damage evolution independent of strain inbond $\varepsilon_{ij}$.

\item The force (derivative of bond energy $w_{ij}$) in the peridynamic spring along the vector between $i$ and $j$ direction is then
\begin{equation}\label{eqn:newsimpleBond}
f = \boldsymbol{\xi} \cdot \nabla_{{\bf u}_i}w_{ij}=  c \alpha_{ij}\varepsilon_{ij} = c \left(\alpha_{ij}(0) - \int_0^t\exp\left(k\varepsilon_{ij}\right)\alpha_{ij}^{n}(1-\alpha_{ij})^{m}dt'\right)\alpha_{ij}\varepsilon_{ij}
\end{equation}
where $c$ is the normal peridynamic elastic spring stiffness.

\end{itemize}

\end{document}
